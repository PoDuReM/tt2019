\documentclass[10pt,a4paper,oneside]{article}
\usepackage[utf8]{inputenc}
\usepackage[english,russian]{babel}
\usepackage{amsmath}
\usepackage{amsthm}
\usepackage{amssymb}
\usepackage{cmll}
\usepackage{enumerate}
\usepackage{stmaryrd}
\usepackage[left=2cm,right=2cm,top=2cm,bottom=2cm,bindingoffset=0cm]{geometry}
\usepackage{proof}
\usepackage{url}
\usepackage{CJKutf8}
\newcommand{\gq}[1]{\texttt{<<}#1\texttt{>>}}
\newcommand{\ogq}[1]{\overline{\texttt{<<}#1\texttt{>>}}}
\begin{document}

\begin{center}{\Large\textsc{\textbf{Теоретические (``малые'') домашние задания}}}\\
             \it Теория типов, ИТМО, М3334-М3339, осень 2019 года\end{center}

\section*{Домашнее задание №1: <<знакомство с лямбда-исчислением>>}

\begin{enumerate}
\item Расставьте скобки:

\begin{enumerate}
\item $\lambda x.x\ x\ \lambda x.x\ x$
\item $(\lambda x.x\ x)\ \lambda x.x\ x$
\item $\lambda x.(x\ x)\ \lambda x.x\ x$
\item $\lambda f.\lambda x.f f f x$
\end{enumerate}

\item Проведите бета-редукции и приведите выражения к нормальной форме:

\begin{enumerate}
\item $(\lambda a.\lambda b.a)\ (\lambda a.\lambda b.a)\ (\lambda a.\lambda b.b)$
\item $(\lambda a.\lambda b.a)\ b$
\item $(\lambda f.\lambda x.f\ (f\ x))\ (\lambda f.\lambda x.f\ (f\ x))$
\end{enumerate}

\item Выразите следующие функции в лямбда-исчислении:

\begin{enumerate}
\item Or, Xor;
\item тернарная операция в Си (\verb!?:!);
\item isZero (T, если аргумент равен 0, иначе F);
\item isEven (T, если аргумент чётный);
\item умножение на 2;
\item умножение;
\item возведение в степень;
\item упорядоченная пара. К паре должны прилагаться три лямбда-выражения ($M$, $P_l$, $P_r$):
выражение $M$ по двум значениям строит упорядоченную пару, а выражения $P_l$ и $P_r$ возвращают
первый и второй элемент упорядоченной пары соответственно.

Убедитесь, что для ваших выражений выполнено $$P_l\ (M\ A\ B) \rightarrow_\beta \dots \rightarrow_\beta A$$ и 
$$P_r\ (M\ A\ B) \rightarrow_\beta \dots \rightarrow_\beta B$$
\item вычитание 1;
\item вычитание;
\item сравнение (<<меньше>>);
\item деление.
\end{enumerate}

\item Назовём бета-эквивалентностью транзитивное, рефлексивное и симметричное замыкание отношения
бета-редукции, будем записывать его как $(=_\beta)$. В частности, бета-эквивалентны те термы,
которые имеют одинаковую нормальную форму.
Также, нетрудно заметить следующее:
\begin{enumerate}
\item $And\ T\ F =_\beta F$;
\item $\Omega =_\beta \Omega$;
\item $(\lambda n.\lambda f.\lambda x.n\ f\ (f\ x))\ \overline{n\vphantom{n+1}} =_\beta \overline{n+1}$;
\item $a \ne_\beta b$.
\end{enumerate}

Мы будем говорить, что лямбда-выражение $E$ выражает функцию $f(x_1,\dots,x_k): \mathbb{N}_0^k \rightarrow \mathbb{N}_0$,
если при любых $x_1,\dots,x_k \in \mathbb{N}_0$
выполнено

$$E\ \overline{x_1}\ \dots\ \overline{x_k} =_\beta \overline{f(x_1,\dots,x_k)}$$

Какие функции выражают следующие выражения? Ответ обоснуйте.
\begin{enumerate}
\item $\lambda m.\lambda n.n\ m$;
\item $\lambda m.\lambda n.\lambda x.n\ (m\ x)$.
\end{enumerate}

\item \emph{Ненормализуемым} назовём лямбда-выражение, не имеющее нормальной формы,
то есть выражение, для которого нет конечной последовательности бета-редукций,
приводящей к нормальной форме.
\emph{Сильно нормализуемым} назовём лямбда-выражение, для которого не существует бесконечной
последовательности бета-редукций (любая последовательность бета-редукций неизбежно 
заканчивается нормальной формой, если её продолжать достаточно долго).
\emph{Слабо нормализуемым} назовём лямбда-выражение, которое имеет нормальную форму,
но существует бесконечная последовательность бета-редукций, которая не приводит его
в нормальную форму. Приведите примеры сильно нормализуемого, слабо нормализуемого и 
ненормализуемого лямбда-выражения.
\end{enumerate}

\end{document}
